\documentclass[12pt]{article}

\usepackage{amsmath}
\usepackage{amssymb}
\usepackage{graphicx}
\usepackage{algorithm}% http://ctan.org/pkg/algorithms
\usepackage{algpseudocode}% http://ctan.org/pkg/algorithmicx
\usepackage{framed} % or, "mdframed"
\usepackage[framed]{ntheorem}
\usepackage{listings}
\newframedtheorem{frm-thm}{Lemma}

\usepackage[scale=1.0, left=1.0cm, right=1.0cm, top=1.865cm, bottom=0.865cm]{geometry}
\DeclareMathOperator*{\argmin}{arg\,min}

\pagestyle{myheadings}
\markright{CSE 260, Assignment 1 \hfill Andrew Conegliano, Matthias Springer\hfill}

\begin{document}

\title{Double-precision General Matrix Multiply (DGEMM)  \\ \vspace{2 mm} {\large Parallel Computation (CSE 260), Assignment 1}}
%\subtitle{Parallel Computation (CSE 260), Assignment 1}
\date{\today}
\author{Andrew Conegliano \and Matthias Springer}
\maketitle

\section{Introduction}

\subsection{Assumptions}
\begin{itemize}
	\item All matrices are square matrices.
	\item All matrices consist of double-precision floating point values (64-bit doubles).
\end{itemize}

\subsection{Notation}
In this work, we use the following notation and variable names.
\begin{itemize}
	\item $n$ is the size of one dimension of the matrices involved. I.e., every matrix has $n^2$ values.
	\item When referring to matrices, $A$ and $B$ denote the source matrices and $C$ denotes the target matrix, i.e. $AB = C$.
\end{itemize}

\section{Hardware}

\section{Basic Algorithm}
The naive implementation of the matrix multiplication algorithm consists of three nested \lstinline{for} loops, where every loop runs from $1$ to $n$. In the innermost loop, a single addition and multiplication is done. Therefore, the runtime complexity of this algorithm is $\mathcal{O}(n^3)$ floating-point operations. We did not use an algorithm with a lower runtime complexity\footnote{Strassen's algorithm has a runtime complexity of $\mathcal{O}(n^{2.8})$.} because the idea of this assignment is to get familiar with memory access and caching. 

\section{Optimizations}

\subsection{Blocking for L1 Cache}

\subsection{Blocking for L2 Cache}
% theoretical explanation why it does not work

\subsection{Matrix Transposition}

\subsection{Register Blocking and Loop Unrolling}

\subsection{Matrix Buffering}

\subsection{Streaming SIMD Extensions (SSE)}

\subsection{Parameter Tuning}

\section{Performance Evaluation}

\end{document}

