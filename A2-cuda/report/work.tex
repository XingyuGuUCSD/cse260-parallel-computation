\documentclass[12pt]{article}

\usepackage{amsmath}
\usepackage{amssymb}
\usepackage{graphicx}
\usepackage{algorithm}% http://ctan.org/pkg/algorithms
\usepackage{algpseudocode}% http://ctan.org/pkg/algorithmicx
\usepackage{framed} % or, "mdframed"
\usepackage[framed]{ntheorem}
\usepackage{listings}
\usepackage{color}
\usepackage[usenames,dvipsnames,svgnames,table]{xcolor}

\newframedtheorem{frm-thm}{Lemma}

\usepackage[scale=1.0, left=1.0cm, right=1.0cm, top=1.865cm, bottom=0.865cm]{geometry}
\DeclareMathOperator*{\argmin}{arg\,min}

\definecolor{dkgreen}{rgb}{0,0.6,0}
\definecolor{dred}{rgb}{0.545,0,0}
\definecolor{dblue}{rgb}{0,0,0.545}
\definecolor{lgrey}{rgb}{0.9,0.9,0.9}
\definecolor{gray}{rgb}{0.4,0.4,0.4}
\definecolor{darkblue}{rgb}{0.0,0.0,0.6}
\lstset{ 
      backgroundcolor=\color{lgrey},  
      basicstyle=\footnotesize \ttfamily \color{black} \bfseries,   
      breakatwhitespace=false,       
      breaklines=true,               
      captionpos=b,                   
      commentstyle=\color{dkgreen},   
      deletekeywords={...},          
      escapeinside={\%*}{*)},                  
      frame=single,                  
      keywordstyle=\color{purple},  
      morekeywords={BRIEFDescriptorConfig,string,TiXmlNode,DetectorDescriptorConfigContainer,istringstream,cerr,exit}, 
      identifierstyle=\color{black},
      stringstyle=\color{blue},      
      language=C,                
      numbers=right,                 
      numbersep=5pt,                  
      numberstyle=\tiny\color{black}, 
      rulecolor=\color{black},        
      showspaces=false,               
      showstringspaces=false,        
      showtabs=false,                
      stepnumber=1,                   
      tabsize=5,                     
      title=\lstname,                 
    }


\pagestyle{myheadings}
\markright{CSE 260, Assignment 1 \hfill Andrew Conegliano, Matthias Springer\hfill}

\begin{document}

\title{Double-precision Matrix Multiply on CUDA  \\ \vspace{2 mm} {\large Parallel Computation (CSE 260), Assignment 2}}
\date{\today}
\author{Andrew Conegliano (A53053325) \and Matthias Springer (A99500782) \and GID G-338-665}
\maketitle

\subsection{Assumptions}
The following assumptions apply within this work.

\begin{itemize}
	\item All matrices are square matrices.
	\item All matrices consist of double-precision floating point values (64-bit doubles).
\end{itemize}

\subsection{Notation}
%\end{tabular}
In this work, we use the following notation and variable names.
\begin{itemize}
	\item $n$ is the size of one dimension of the matrices involved. I.e., every matrix has $n^2$ values.
	\item When referring to matrices, $A$ and $B$ denote the source matrices and $C$ denotes the target matrix, i.e. $AB = C$.
	\item $b_i$, $b_j$ and $b_k$ denote the block size for each \emph{dimension} $i$, $j$, and $k$, respectively\footnote{We tried multiple levels of blocking and it is evident from the context which level of blocking we are refering to.}.
\end{itemize}

\section*{Runtime Environment}
We optimized our implementation for a specific runtime environment. All benchmarks and performance results are based on the following hardware and software.

\subsection{Hardware (Dirac)}
\begin{itemize}
	\item 2 Nahalem Quad Core CPUs
	\item 24 GB DDR3 RAM
	\item 1 Tesla C2050 (Fermi) GPU
	\begin{itemize}
		\item 14 vector units $\times$ 32 cores = 448 total cores \@ 1.15 GHz
		\item 3 GB DDR5 RAM, 2.625 GB usable due to ECC
	\end{itemize}
\end{itemize}

\subsection{Software}

\section{Basic Algorithm}

\begin{figure}
\begin{lstlisting}
for i = 0 .. n - 1
	for j = 0 .. n - 1
		cij = C[i*n + j]

		for k = 0 .. n - 1
			cij += A[i*n+k] * B[k*n+j]

		C[i*n+j] = cij
\end{lstlisting}
\caption{Naive matrix multiplication algorithm.}
\label{fig:naive_mul}
\end{figure}

\end{document}

